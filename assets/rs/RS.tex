\documentclass[12pt]{article}

\usepackage{fullpage}
\usepackage{url}
\usepackage{natbib}
\usepackage{xcolor}
\usepackage{amsmath}
\usepackage{amssymb}
\usepackage{amsthm}
\usepackage{svg}
\usepackage{hyperref}

\newcommand{\GQ}{G_{\mathbb{Q}}}
\newcommand{\newstuff}[1]{\color{blue}{#1}\color{black}}

\newtheorem{theorem}{Theorem}
\newtheorem*{theorem*}{Theorem}

% \pagestyle{empty}

\begin{document}

\noindent Seewoo Lee \\
Research Statement \\
% \today
\bigskip



My research centers on number theory, particularly delving into the realms of automorphic forms and the (relative) Langlands program, leveraging computational tools to enhance exploration and understanding.


\section*{Relative Langlands program}

Introduced by Robert Langlands, the \emph{Langlands Program} constitutes a comprehensive unification theory in number theory and beyond, seeking to establish connections between \emph{automorphic forms / representations} and \emph{Galois representations}.
Specifically, for an ``algebraic'' automorphic representation $\pi$ of a group $G(\mathbb{A}_{\mathbb{Q}})$, it is conjectured that there exists an associated Galois representation $\rho_{\pi}: \mathrm{Gal}(\overline{\mathbb{Q}} / \mathbb{Q}) \to \check{G}(k)$, where $k$ can be $\mathbb{C}$ or $\overline{\mathbb{Q}_{\ell}}$, such that the automorphic $L$-function $L(s, \pi)$ coincides with the Artin $L$-function $L(s, \rho_{\pi})$. Here, $\check{G}$ denotes the Langlands dual group of $G$. The case where $G = \mathrm{GL}_{1}$ is well understood through class field theory, and for $G = \mathrm{GL}_{2}$, the conjecture is proven when $\pi$ is associated with holomorphic modular forms, culminating in the celebrated modularity theorem by Wiles, Taylor, and others, which played a key role in the proof of Fermat’s Last Theorem.

Recently, there has been growing interest in the \emph{relative} Langlands program, which seeks to extend the classical Langlands program to spherical varieties \cite{sakellaridis2012spherical,sakellaridis2017periods}.
A major goal of this program is to systematically address the many instances of identities between automorphic periods and special values of $L$-functions (Ichino--Ikeda type formula \cite{ichino2010periods}), as seen in the works of Hecke, Iwasawa--Tate, Rankin--Selberg, Godement--Jacquet, Ichino--Ikeda, and Lapid--Mao. 
Moreover, Ben-Zvi, Sakellaridis, and Venkatesh have put forward a conjectural \emph{duality} within the relative Langlands program \cite{ben2023relative}.
This duality suggests that for a Hamiltonian $G$-space $G \curvearrowright M$, one should expect a corresponding dual pair $\check{G} \curvearrowright \check{M}$, where a \emph{dual identity} between automorphic periods and special values of $L$-functions arises by exchanging the roles of $(G, M)$ and $(\check{G}, \check{M})$.


\subsection*{Mao--Rallis trace formula for dual pairs}

I am working on a conjectural identity relating automorphic periods and $L$-values for the dual pairs introduced by Mao and Rallis \cite{mao1997trace}.
For simple, split, and simply-laced groups $G'$, excluding type $A_n$, Mao and Rallis construct a dual pair $(\mathrm{SL}_{2}, G)$ within $G'$ by utilizing Heisenberg groups associated with $G'$.
This includes cases such as $(\mathrm{SL}_{2}, \mathrm{SL}_{6})$ within $E_6$.
They proposed conjectural relative trace formulae for these dual pairs, suggesting that period integrals of automorphic forms on $G$ (resp. $\mathrm{SL}_{2}$) should correspond to special $L$-values associated with automorphic representations of $\mathrm{SL}_{2}$ (resp. $G$).
In their work, they established the fundamental lemma for unit elements in the Hecke algebras of both $G$ and $\mathrm{SL}_{2}$.

Recently, Mao, Wan, and Zhang \cite{mao2023bzsv} formulated a refined version of this conjecture, in the form of an Ichino--Ikeda type formula, within the framework of the relative Langlands program \cite{ben2023relative}.
They also proved smooth transfer for local functions at non-Archimedean places.
In collaboration with Yuchan Lee, we are working to complete the comparison of the relative trace formula for the case $(\mathrm{SL}_{2}, \mathrm{SL}_{6})$, focusing on proving the fundamental lemma for the full Hecke algebras, establishing smooth transfer at Archimedean places, and analyzing the spectral decomposition of the relative traces.
Together with the result of Lapid--Mao \cite{lapid2015conjecture} on Whittaker--Fourier coefficients, this would confirm one direction of the non-refined conjecture: if $\Pi$ is a cuspidal automorphic representation of $\mathrm{SL}_{2}$ and $\pi$ is its functorial lift to $G$ via the $L$-group morphism $\mathrm{PGL}_{2} \to {}^L G$, then the Mao--Rallis automorphic period is nonzero if the adjoint $L$-value $L(1, \Pi, \mathrm{Ad})$ is nonzero.
Additionally, we aim to prove the refined formula by establishing local relative character identities and combining them with the refined identity of Lapid--Mao for $\mathrm{SL}_{2}$.
It is worth noting that this refined formula is "dual" to the conjectural formula for the Ginzburg--Rallis periods and the exterior cube $L$-values of $\mathrm{PGL}_{6}$ \cite{ginzburg2000exterior}.


\subsection*{Ichino--Ikeda formula for general spin groups (Bessel case)}

Building upon the groundwork laid by Liu \cite{liu2016refined} on the special orthogonal groups ($\mathrm{SO}_{2} \times \mathrm{SO}_{5}$ and $\mathrm{SO}_{3} \times \mathrm{SO}_{6}$) and drawing insights from Emory's work \cite{emory2020global} on \emph{general spin groups} ($\mathrm{GSpin}_{n} \times \mathrm{GSpin}_{n+1}$ for $n = 2, 3, 4$), I am working on the Ichino--Ikeda conjecture for general spin groups, particularly in cases involving general Bessel periods.
Furthermore, I'm trying to generalize Furusawa and Morimoto's work on the $\mathrm{SO}_{2} \times \mathrm{SO}_{2n+1}$ case and B\"ocher's conjecture \cite{furusawa2020refined} in this direction.
My approach involves leveraging exceptional isomorphisms between low-rank general spin groups and other classical groups and reducing the conjecture to the already known cases.


\section*{Computational Approach in Number Theory}

In the realm of the Langlands Program, dealing with abstract objects like Galois representations and automorphic forms often benefits from grounding these concepts in tangible, computable counterparts.
Especially, these ``classical'' objects (e.g. modular forms and Maass wave forms, instead of automorphic representations of $\mathrm{GL}_{2}(\mathbb{A}_{\mathbb{Q}})$) are usually easy to compute explicitly with help of computer algebra systems like SageMath \cite{sagemath} or MATLAB \cite{MATLAB}.


\subsection*{Modular forms and optimal sphere packings}

The \emph{optimal sphere packing problem} seeks the densest arrangement of unit balls in $d$-dimensional space $\mathbb{R}^{d}$.
While the problem is trivial for $d = 1$, the two-dimensional case was solved by Thue in 1890.
The three-dimensional case, known as Kepler's conjecture, was settled by Thomas Hales in 2005 using extensive computer calculations \cite{hales2005proof}, with the formal verification completed nearly a decade later through the use of proof assistants HOL Light and Isabelle \cite{hales2017formal}.

An unexpected link between the $8$- and $24$-dimensional sphere packing problems and number theory emerged through the work of Cohn and Elkies, who introduced the \emph{linear programming bound} \cite{cohn2003new}. 
This approach hinted that finding specific ``magic functions'' could unlock optimal sphere packings in these dimensions.
However, constructing such functions, which must satisfy constraints on both the function and its Fourier transform, is challenging due to the uncertainty principle.
Maryna Viazovska made a breakthrough by using modular forms to construct the magic function for dimension 8 \cite{viazovska2017sphere}, and similar methods soon resolved the case for dimension 24 \cite{cohn2017sphere}.

To prove these two cases, the authors \cite{viazovska2017sphere,cohn2017sphere} relied on numerical approximations and extensive computer assisted computations to establish desired inequalities between modular forms, and it is natural to ask if there is a more general and conceptual proof for these inequalities.
While a more direct proof exists for the dimension 8 case by Dan Romik \cite{romik2023viazovska}, I have found \emph{algebraic} proofs for both the 8- and 24-dimensional cases that avoid reliance on numerical approximations \cite{lee2024algebraic}.
In my work, I developed a theory of \emph{positive} and \emph{completely positive} quasimodular forms, which I used to study the \emph{magic modular forms} appearing in the optimal sphere packing problem. 
Additionally, I discovered connections to Kaneko and Koike's \emph{extremal quasimodular forms} \cite{kaneko2006extremal}, which are conjectured to have non-negative Fourier coefficients.
A key aspect of my approach involves leveraging the differential equations satisfied by these modular forms.
My work opens new possibilities for generalizing Viazovska's construction to dimensions beyond 8 and 24, based on Fourier eigenfunctions constructed by Feigenbaum, Grabner, and Hardin \cite{feigenbaum2021eigenfunctions}.
In particular, this could lead to a new upper bound for the uncertainty principle in specific dimensions \cite{bourgain2010principe}.
Furthermore, as a byproduct of my research, I proved Kaneko and Koike's conjecture regarding the positivity of Fourier coefficients for extremal forms in the case of depth 1 \cite{kaneko2006extremal}.

These proofs were inspired by extensive experiments using SageMath. 
Notably, after observing the plot of the quotient of two modular forms (Figure \ref{fig:d8graph}), I realized the key properties to prove - monotonicity and the limit as $t \to 0^+$ - both of which turned out to hold true (Propositions 5.1 and 5.2 of \cite{lee2024algebraic}).

\begin{figure}[h]
    \centering
    \includesvg[width=0.6\textwidth]{./d8.svg}
    \caption{Graph of the quotient $F(it)/G(it)$ of two modular forms as a function in $t > 0$.}
    \label{fig:d8graph}
\end{figure}
\subsection*{Maass wave forms, Quantum modular forms, and Hecke operators}


In \cite{cohen1988q}, Cohen constructed the first explicit example of a Maass wave form, based on one of Ramanujan's $q$-series. The coefficients of this form are related to a Hecke character of the real quadratic field $\mathbb{Q}(\sqrt{6})$, and Cohen conjectured that this Maass wave form is an eigenform for suitable Hecke operators. However, the usual Hecke operators are not appropriate in this case, as the multiplier system (Nebentypus) of Cohen's Maass wave form does not arise from Dirichlet characters.

In my undergraduate and master's thesis, I proposed a correct definition of Hecke operators that applies to more general multiplier systems, including Cohen's Maass wave form. I further proved that this Maass wave form is indeed an eigenform under these operators \cite{lee2018quantum,lee2019maass}. Additionally, one can associate \emph{quantum modular forms} to the Maass wave form via period integrals, following the work of Lewis and Zagier \cite{lewis2001period,zagier2010quantum}, and I demonstrated that this map is Hecke-equivariant. As a result, this leads to nontrivial identities between certain $p$-th roots of unity and the $p$-th coefficients of the Maass wave form for primes $p$. The same argument applies to the Maass wave form of Li, Ngo, and Rhoades \cite{li2013renormalization}.

% \newpage
\section*{Other Projects}

My interest is not restricted to number theory. I'm interested in various subjects, including
\begin{itemize}
    \item formalization of mathematics,
    \item discrete geometry,
    \item homomorphic encryption.
\end{itemize}




\subsection*{Formalization of polynomial FLT and sphere packing in $\mathbb{R}^{8}$}

The formal verification of mathematical proofs is a rapidly growing field aimed at ensuring the correctness of mathematical results.
A number of significant mathematical objects and proofs have been formalized, including Hales' proof of Kepler's conjecture \cite{hales2017formal}, schemes \cite{buzzard2022schemes}, perfectoid spaces \cite{buzzard2020formalising}, and Gowers, Green, Manners, and Tao's proof of the polynomial Freiman--Ruzsa conjecture \cite{gowers2023conjecture,pfr}.

One notable ongoing project is the formalization of the proof of Fermat's Last Theorem in Lean 4, led by Kevin Buzzard \cite{fltlean4}.
Given the complexity of the proof and the advanced mathematics required, many of which are not yet in Lean's \texttt{mathlib4} library \cite{mathlib4}, it is estimated that the complete formalization could take over 10 years.
In contrast, the \emph{polynomial} version of Fermat's Last Theorem is much simpler to prove, and a more general result, known as the Mason--Stothers theorem, provides an analogue of the ABC conjecture for polynomials \cite{stothers81,mason84}.

In collaboration with Jineon Baek, we give a complete formalization of the Mason--Stothers theorem in Lean 4 \cite{baek2024formalizing}.
While the theorem has previously been formalized in HOL by Eberl \cite{eberl17} and in Lean 3 by Wagemaker \cite{wagemaker18}, our work provides a detailed comparison with these prior formalizations \cite[Section 7]{baek2024formalizing}.
We are also in the process of integrating our formalization into the \texttt{mathlib4} library, with the code available at:

\begin{center} \href{https://github.com/seewoo5/lean-poly-abc}{https://github.com/seewoo5/lean-poly-abc} \end{center}


Additionally, I am working on a project to formalize Viazovska's proof of the optimal sphere packing in $\mathbb{R}^{8}$ \cite{viazovska2017sphere} in Lean 4, alongside a team led by Sidharth Hariharan, which includes Chris Birkbeck, Gareth Ma, and Maryna Viazovska.
This effort may also involve formalizing my algebraic proof of the modular form inequalities \cite{lee2024algebraic}, which would allow us to bypass the need for formalizing various aspects of numerical analysis and the Hardy--Ramanujan formula.
So far, we have completed the formalization of the $E_8$ lattice and its density, and are now working on formalizing the Cohn--Elkies bound and the foundational theory of (quasi)modular forms.


\subsection*{Conway--Soifer conjecture - homothetic case}

Consider an equilateral triangle with side length $n + \varepsilon$, where $n \ge 1$ is an integer and $\varepsilon > 0$ is sufficiently small.
What is the minimum number of unit equilateral triangles required to cover this larger triangle?
By considering the area, it is straightforward to show that at least $n^2 + 1$ unit triangles are necessary.
Conway and Soifer provided two different ways to cover the large triangle using $n^2 + 2$ unit triangles \cite{conway2005covering} and conjectured that this is the minimum number needed.

In collaboration with Jineon Baek, we proved that this conjecture holds if we restrict the covering to \emph{homothetic} triangles, i.e., when the sides of the unit triangles are parallel to those of the large triangle (aligned either as $\bigtriangleup$ or $\bigtriangledown$) \cite{baek2024n2}.
Specifically, we established the following general result:

\begin{theorem*}[Baek--L.] A triangle is called a horizontal triangle of base $b$ and height $h$ if one of its sides of length $b$ is parallel to the $x$-axis, and its height $h$ is measured along the $y$-axis. Then $n^2 + 1$ horizontal triangles of base $b$ and height $h$ cannot cover a horizontal triangle of base $nb$ and height greater than $nh$. \end{theorem*}

Our proof is elementary, and we also determined the largest possible values of $\varepsilon$ such that an equilateral triangle of side length $n + \varepsilon$ can be covered by either $n^2 + 2$ or $n^2 + 3$  homothetic unit triangles.
Specifically, these values are $\varepsilon= 1 / (n + 1)$ and $\varepsilon = 1 / n$, respectively.
We believe that our method can be generalized to higher dimensions or extended to determine the largest side length of an equilateral triangle that can be covered by $n^2 + k$ homothetic unit triangles for $1 \le k \le 2n$.


\subsection*{Encrypted transfer learning with homomorphic encryption}

During my alternative military service as a Research Engineer at CryptoLab, I developed a privacy-preserving machine learning library called \texttt{HEaaN.SDK} \cite{heaansdk}, based on the Cheon--Kim--Kim--Song (CKKS) homomorphic encryption (HE) scheme \cite{cheon2017homomorphic}.
The CKKS scheme theoretically allows arbitrary arithmetic computations on encrypted real and complex numbers (with small errors), which might lead one to believe that implementing machine learning algorithms using HE is straightforward.
However, encrypted computations are significantly slower than plaintext computations, and naive implementations can be highly impractical due to performance bottlenecks.
To address this, algorithms need to be redesigned in an \emph{HE-friendly} way, which is often a complex research problem.

One key challenge I encountered was the lack of HE-based algorithms for \emph{multiclass} classification tasks; most prior work focused on binary classifications with a limited number of features.
Implementing a multiclass classification algorithm using HE required overcoming two major obstacles: (1) efficiently performing encrypted softmax computations with large inputs and (2) executing large-scale encrypted matrix multiplications.

These challenges were addressed in HETAL (\textbf{H}omomorphic \textbf{E}ncryption-based \textbf{T}ransfer \textbf{L}earning) \cite{lee2023hetal}. 
For softmax computation, we used homomorphic comparison techniques \cite{cheon2020efficient} to normalize inputs by subtracting the maximum value, followed by homomorphic domain extension \cite{cheon2022efficient}, which significantly reduced errors and expanded the input range compared to previous approaches \cite{jin2020secure,lee2022privacy,hong2022secure}.
To optimize matrix multiplications, we implemented two distinct multiplication methods: $AB^\intercal$ and $A^\intercal B$ which allowed us to bypass the costly transpose operation.
We further employed tiling and complex packing techniques to minimize the number of required rotations, resulting in matrix multiplication algorithms that were 1.8 to 323 times faster than previous methods \cite{crockett2020low,jin2020secure}.
As a result, we successfully fine-tuned commonly used vision and language models within an hour on five benchmark datasets, using a single A40 GPU. 
This demonstrated that HE-based encrypted fine-tuning is not only feasible but also practical for real-world applications.




\bibliographystyle{plain} % We choose the "plain" reference style
{
\scriptsize
\bibliography{refs} % Entries are in the refs.bib file
}

\end{document}